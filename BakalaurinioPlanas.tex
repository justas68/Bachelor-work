\documentclass{VUMIFPSbakalaurinis}
\usepackage{algorithmicx}
\usepackage{algorithm}
\usepackage{algpseudocode}
\usepackage{amsfonts}
\usepackage{amsmath}
\usepackage{bm}
\usepackage{caption}
\usepackage{color}
\usepackage{float}
\usepackage{graphicx}
\usepackage{listings}
\usepackage{subfig}
\usepackage{wrapfig}

\university{Vilniaus universitetas}
\faculty{Matematikos ir informatikos institutas}
\department{Programų sistemų katedra}
\papertype{Bakalauro darbas}
\title{Gestų kalbos vienetų atpažinimas iš video srauto}
\titleineng{Recognition of Sign language units from a video stream}
\author{Pranciškus Ambrazas}
\supervisor{j. asist. Linas Petkevičius}
\reviewer{dr. Vytautas Valaitis}
\date{Vilnius – \the\year}

\begin{document}
\maketitle
\section{test}
tes
\printbibliography[heading=bibintoc]  % Šaltinių sąraše nurodoma panaudota
% literatūra, kitokie šaltiniai. Abėcėlės tvarka išdėstomi darbe panaudotų
% (cituotų, perfrazuotų ar bent paminėtų) mokslo leidinių, kitokių publikacijų
% bibliografiniai aprašai. Šaltinių sąrašas spausdinamas iš naujo puslapio.
% Aprašai pateikiami netransliteruoti. Šaltinių sąraše negali būti tokių
% šaltinių, kurie nebuvo paminėti tekste. Šaltinių sąraše rekomenduojame
% necituoti savo kursinio darbo, nes tai nėra oficialus literatūros šaltinis.
% Jei tokių nuorodų reikia, pateikti jas tekste.

\appendix

\section{testS}
test
\end{document}
